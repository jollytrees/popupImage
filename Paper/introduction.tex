\section{Introduction} \label{introduction}
Popup craft, introduced by Masahiro Chatani in 1980, is a design of cuts and folds on a single piece of paper. Complex and interesting structure pops up when the paper is opened. The special structure of a popup craft makes it appealing but hard to design. In this work, we seek an automatic design method via an optimization process. 

Given either a Vector image or a scalar image, we first segment it to get a segmentation image $\mathcal{S}$ which stores a segment index for each pixel. Based on $\mathcal{S}$, we find initial fold lines between pair of different segments. Given these initial fold lines, we find other fold line candidates in a sense that a new fold line candidate affects the topology of initial fold lines. With all the fold lines as candidates, we determine their activeness, convexity, and positions in an optimization framework. By adding auxiliary variables and constraints, we enforce an optimal solution to be a popup craft design with satisfactory properties. First, a popup craft must be foldable, which means it can be opened and folded without breaking its geometry (see \cite{li2010popup} for more details). Second, a popup craft is more appealing when it is stable, which means the structure is stable when two border of the paper is holded (see \cite{li2010popup} for more details). Third, because the number of segments of the popup craft is limited in our image-based problem setting, the connectivity, which enforces there is only one connected structure in the popup craft, becomes important. Fourth, the popup craft should follow the input shape.